Interacting with the touchscreen on the V5 Brain is made possible through \href{https://littlevgl.com}{\tt L\+V\+GL}. L\+V\+GL is a full-\/featured C graphics library (it\textquotesingle{}s accessible in C++ projects too under the same A\+PI).

The first step to getting started with L\+V\+GL is to include {\ttfamily \hyperlink{apix_8h}{pros/apix.\+h}} in your {\ttfamily \hyperlink{main_8h}{main.\+h}} file or other header files. This includes the full L\+V\+GL feature set as described in their documentation\+: \href{https://littlevgl.com/}{\tt https\+://littlevgl.\+com/}

You can follow along with any of the L\+V\+GL \href{https://github.com/littlevgl/lv_examples/tree/master/lv_tutorial}{\tt tutorials} or \href{https://docs.littlevgl.com/#Objects}{\tt wiki}. There is no need to port or initialize L\+V\+GL, you can simply start creating objects.

\begin{quote}
{\bfseries note}

Custom L\+V\+GL code cannot be displayed at the same time as the \href{./llemu.html}{\tt L\+L\+E\+MU}. \+: As a result, you must remove the L\+L\+E\+MU code ({\ttfamily \hyperlink{namespacepros_1_1lcd}{pros\+::lcd}}) that is present in {\ttfamily initialize.\+cpp} by default in a new project.\end{quote}
