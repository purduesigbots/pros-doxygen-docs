P\+R\+OS projects are internally composed of three parts\+: the {\bfseries P\+R\+OS library} (found in {\ttfamily /firmware}), the {\bfseries header} files (found in {\ttfamily /include}), and {\bfseries user code} (found in {\ttfamily /src}).

\subsection*{firmware}

The {\bfseries P\+R\+OS library} is a single file containing the core P\+R\+OS routines. This file does not need to be changed. If there appears to be an issue with a core P\+R\+OS function, please file an issue on \href{https://github.com/purduesigbots/pros/issues}{\tt Git\+Hub}.

\subsection*{include}

The {\bfseries header} files are all found in the {\ttfamily include} directory. One header file, \href{../../api/index.html}{\tt api.\+h}, is required to declare the P\+R\+OS library functions. This file exists merely to include the individual P\+R\+OS A\+PI headers, all of which can be found in {\ttfamily include/pros}. Each header file in this directory covers a specific aspect of interacting with the V5 hardware, and correlates to the files found in the \href{../../api/index.html}{\tt A\+PI documentation}.

The other file, {\ttfamily \hyperlink{main_8h}{main.\+h}}, is intended for declaring functions and variables shared between the user code files. {\ttfamily \hyperlink{main_8h}{main.\+h}} also offers a variety of configurable options for tailoring P\+R\+OS to your needs.


\begin{DoxyItemize}
\item {\ttfamily P\+R\+O\+S\+\_\+\+U\+S\+E\+\_\+\+S\+I\+M\+P\+L\+E\+\_\+\+N\+A\+M\+ES}\+: If defined, some commonly used enums will have preprocessor macros which give a shorter, more convenient naming pattern. For instance, E\+\_\+\+C\+O\+N\+T\+R\+O\+L\+L\+E\+R\+\_\+\+M\+A\+S\+T\+ER has a shorter name\+: {\ttfamily C\+O\+N\+T\+R\+O\+L\+L\+E\+R\+\_\+\+M\+A\+S\+T\+ER}. {\ttfamily E\+\_\+\+C\+O\+N\+T\+R\+O\+L\+L\+E\+R\+\_\+\+M\+A\+S\+T\+ER} is pedantically correct within the P\+R\+OS styleguide, but not convenient for most student programmers.
\item {\ttfamily using namespace pros}\+: This can be uncommented to be added with the use of {\ttfamily P\+R\+O\+S\+\_\+\+U\+S\+E\+\_\+\+S\+I\+M\+P\+L\+E\+\_\+\+N\+A\+M\+ES}. This reduces the length of declarations when using C++, allowing you to simply declare a {\ttfamily Motor} instead of a {\ttfamily \hyperlink{classpros_1_1Motor}{pros\+::\+Motor}}. This will make the code appear cleaner and will be simpler for newer programmers, but is typically considered \href{https://msdn.microsoft.com/en-us/library/5cb46ksf.aspx}{\tt bad practice}. As a result, this line is commented out by default.
\end{DoxyItemize}

New header files can be created in the include directory, as long as the name ends with {\ttfamily .h} (Traditionally for C files) or {\ttfamily .hpp} (for C++ files). See this \href{http://www.learncpp.com/cpp-tutorial/19-header-files/}{\tt C++ tutorial} for more information on how to create header files.

\subsection*{src}

{\bfseries User code} has the actual sequential instructions that govern the robot\textquotesingle{}s behavior. Prior to P\+R\+OS kernel 3.\+2.\+0, new projects by default split user code into autonomous ({\ttfamily autonomous.\+c} or {\ttfamily autonomous.\+cpp}), driver control ({\ttfamily opcontrol.\+c} or {\ttfamily opcontrol.\+cpp}), and initialization ({\ttfamily initialize.\+c} or {\ttfamily initialize.\+cpp}) files. Code in one file can talk to code in another file using declarations in the header files. Beginning with P\+R\+OS kernel 3.\+2.\+0, new projects by default have a single {\ttfamily main.\+cpp} file that contains all of the competition task functions.

New user code files can be created in the {\ttfamily src} directory, as long as the name ends with {\ttfamily .c} or {\ttfamily .cpp} it will be compiled with the others.

All user code files should start with\+: \begin{DoxyVerb}#include "main.h"
\end{DoxyVerb}


This will ensure that the P\+R\+OS A\+PI and other critical definitions are available in each file.

While more complicated than some environments, splitting up code grants powerful modularity and code reusability, especially when combined with source control. 