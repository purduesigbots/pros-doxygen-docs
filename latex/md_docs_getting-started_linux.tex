The steps for installing the toolchain can differ greatly per distribution. The first step is to check whether your package manager offers the arm-\/none-\/eabi toolchain. If so, make sure that the version available is greater than or equal to 7.\+2 before installing.

If you\textquotesingle{}re not sure whether your distribution\textquotesingle{}s package manager has the toolchain available, or if you prefer to install things manually, follow the instructions below.

\begin{quote}
{\bfseries note}

For users of Debian-\/based distributions, be aware that the toolchain available through Apt is out of date and likely will not work for P\+R\+OS projects. For Ubuntu users, you may see references online to a P\+PA by team-\/gcc-\/arm-\/embedded, but that P\+PA does not seem to be updated any more. Therefore, if you are using a Debian-\/based distribution or Ubuntu, please follow the instructions below. \end{quote}



\begin{DoxyEnumerate}
\item Download the latest version of the toolchain from \href{https://developer.arm.com/tools-and-software/open-source-software/developer-tools/gnu-toolchain/gnu-rm/downloads}{\tt the Arm developer site}. We recommend the \char`\"{}\+Linux 64-\/bit\char`\"{} release. If you can\textquotesingle{}t use this for some reason (e.\+g. you have a 32-\/bit system), you may need to download the \char`\"{}\+Source Invariant\char`\"{} release and build manually, following the instructions in the archive.
\item Move the file to your home directory and untar it using the command tar -\/xjvf gcc-\/arm-\/none-\/eabi-\/\+X-\/20\+X\+X-\/q\+X-\/update-\/linux.\+tar.\+bz2\+:. The Xs should be replaced with those present in the name of the file you downloaded.
\item Add this line to your .bashrc file (if using bash), replacing $<$your user$>$ with your username\+: export P\+A\+TH=\$P\+A\+TH\+:/home/$<$your user$>$/gcc-\/arm-\/none-\/eabi-\/8-\/2019-\/q3-\/update/bin/. If you are using a shell other than bash, refer to that shell\textquotesingle{}s documentation for how and where to add entries to your P\+A\+TH when logging in.
\item Close and re-\/open your terminal, or run source/.bashrc (if running bash).
\item Test by running arm-\/none-\/eabi-\/gcc --version. The output should confirm that the version is greater than or equal to 7.\+2. If it is not, make sure you don\textquotesingle{}t have conflicting versions installed through a package manager.
\end{DoxyEnumerate}

\begin{quote}
{\bfseries note}

After installing the toolchain using the instructions listed above, upgrading to a newer version is as simple as removing the previous install and following the instructions again with the newer version. \end{quote}



\begin{DoxyEnumerate}
\item If you do not already have one installed, install a version of Python greater than or equal to 3.\+6
\item Check the latest version of the P\+R\+OS C\+LI on \href{https://github.com/purduesigbots/pros-cli3/releases/latest}{\tt our releases page}, and run python3.\+6 -\/m pip install --user \href{https://github.com/purduesigbots/pros-cli/releases/download/3.X.X/pros_cli_v5-3.X.X-py3-none-any.whl,}{\tt https\+://github.\+com/purduesigbots/pros-\/cli/releases/download/3.\+X.\+X/pros\+\_\+cli\+\_\+v5-\/3.\+X.\+X-\/py3-\/none-\/any.\+whl,} replacing the number after \textquotesingle{}python\textquotesingle{} with the version you installed and the Xs with the numbers you found before. If you wish to install for all users, run the command with sudo and remove the --user flag.
\item Run prosv5 --version to verify the C\+LI was installed correctly. If the command doesn\textquotesingle{}t work, try restarting your machine.
\end{DoxyEnumerate}

\begin{quote}
{\bfseries note}

The following instructions are for installing Atom and cquery. If you intend to use an editor other than Atom, this section is optional. \end{quote}



\begin{DoxyEnumerate}
\item Follow the instructions \href{https://github.com/cquery-project/cquery/wiki/Building-cquery}{\tt here} to build and install cquery.
\item \href{https://atom.io}{\tt Install Atom}.
\item Run apm install \href{mailto:pros-bootstrapper@0.0.12}{\tt pros-\/bootstrapper@0.\+0.\+12}.
\item Open Atom and wait for any plugins to finish installing.
\item Happy coding!
\end{DoxyEnumerate}

\begin{quote}
{\bfseries note}

If Atom seems to get stuck during step 4, restart Atom every few minutes.\end{quote}
