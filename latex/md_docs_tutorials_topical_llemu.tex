\begin{quote}
{\bfseries note}

For a full list of functions for interacting with the L\+L\+E\+MU, see its \+: \href{../../api/c/llemu.html}{\tt C A\+PI} and \href{../../api/cpp/llemu.html}{\tt C++ A\+PI}. \end{quote}


\subsection*{Initialization}

Initialization of the L\+L\+E\+MU is very simple, it\textquotesingle{}s just a call to its initialization function at whatever point in the program you would like the L\+L\+E\+MU to start displaying (this will most likely be in {\ttfamily \hyperlink{main_8h_a9efe22aaead3a5e936b5df459de02eba}{initialize()}}).

Initialization is done as such\+:

\subsection*{Writing to the L\+L\+E\+MU}

Writing to the L\+L\+E\+MU is nearly identical to writing to the L\+CD with \href{../../cortex/tutorials/lcd.html}{\tt P\+R\+OS 2}. Most writing should be done with the print function, which is analogous to \href{http://www.cplusplus.com/reference/cstdio/printf/}{\tt printf}.

\subsection*{Using the Buttons}

Using the buttons can be done in a similar method to \href{../../../cortex/tutorials/lcd.html}{\tt P\+R\+OS 2} with the \href{../../api/cpp/llemu.html#read-buttons}{\tt pros\+::lcd\+::read\+\_\+buttons} function. See the above example for printing the button readings.

While this is sufficient for most applications, some tasks are easier to perform using the \href{../../api/cpp/llemu.html#register-btn0-cb}{\tt pros\+::lcd\+::register\+\_\+btn\#\+\_\+cb} functions (where \# is replaced with 0, 1, or 2 for the left, center, and right buttons respectively). With these function you can assign a function to be called each time that the button is pressed.

\begin{quote}
{\bfseries note}

Custom L\+V\+GL code cannot be displayed at the same time as L\+L\+E\+MU.\end{quote}
