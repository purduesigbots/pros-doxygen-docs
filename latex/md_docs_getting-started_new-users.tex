The most important characteristic of P\+R\+OS to note when getting started is that P\+R\+OS is just standard C or C++ programming. Anything that works in standard C/\+C++ will work as a part of a P\+R\+OS project, and similarly the errors given for code that doesn\textquotesingle{}t work will match errors given for any similar C/\+C++ code. Learning C/\+C++ is essential for using P\+R\+OS.

\subsection*{I\textquotesingle{}ve never used P\+R\+OS or written C/\+C++ code before, how do I start?}

If you have not used P\+R\+OS or done non-\/\+V\+EX C/\+C++ code development before, we recommend that you check out C tutorials on the following topics\+:


\begin{DoxyItemize}
\item \href{http://www.studytonight.com/c/user-defined-functions-in-c.php}{\tt Functions}. C is a language that heavily emphasizes functions, and knowing how they work is essential to using P\+R\+OS. The \href{../api/index.html}{\tt P\+R\+OS A\+PI} is a set of functions, so any time that you want to interact with a sensor or motor, you\textquotesingle{}re using functions.
\item \href{https://www.tutorialspoint.com/cprogramming/c_header_files.htm}{\tt Header Files}. The P\+R\+OS template (the set of files automatically created when you start a P\+R\+OS project) contains a couple of header files, and it\textquotesingle{}s recommended that you make additional header files as you develop your code. Header files contain the declarations for functions and global variables (among other things), which is why the \href{../api/index.html}{\tt P\+R\+OS A\+PI} can be found in {\ttfamily include/pros/api.\+h}. Knowing what code should go in a header file ({\ttfamily .h}, {\ttfamily .hpp}) or a source file ({\ttfamily .c}, {\ttfamily .cpp}) can be difficult to determine at first, but it is a very useful skill to learn.
\item \href{https://www.codingunit.com/printf-format-specifiers-format-conversions-and-formatted-output}{\tt printf()}. At some point when developing P\+R\+OS code, you will likely want to get some feedback on what the value of a variable is. This is not an exact replacement for a full debugging utility by any means, but is the standard method for troubleshooting issues in most languages and can be used for viewing sensor values or your own variables\textquotesingle{} values. The output from these {\ttfamily printf()} statements can be viewed in the terminal by running {\ttfamily pros terminal}.
\item \href{../tutorials/topical/multitasking.html}{\tt Tasks}. One common mistake that new P\+R\+OS users make is forgetting to include a {\ttfamily \hyperlink{rtos_8h_ab8c5a8048d5576a33d7f79b95a2fa0dd}{delay()}} statement in their tasks (this includes {\ttfamily \hyperlink{main_8h_a1903abdb5ef0f301d660754c8315fc17}{opcontrol()}} too), starving the processor of resources and preventing the P\+R\+OS kernel from running properly. Every infinite loop, like the one in {\ttfamily \hyperlink{main_8h_a1903abdb5ef0f301d660754c8315fc17}{opcontrol()}}, needs to have a delay statement. We recommend at least 2ms.
\end{DoxyItemize}

And then for additional C tutorial topics, visit \href{https://www.cprogramming.com/tutorial/c-tutorial.html}{\tt C\+Programming.\+com} or \href{http://www.studytonight.com/c/overview-of-c.php}{\tt Study\+Tonight}. A good video tutorial series (as opposed to the previous text-\/based tutorials) can be found on \href{https://youtu.be/nXvy5900m3M}{\tt You\+Tube}.

\subsection*{I know C/\+C++, now how do I use P\+R\+OS?}

The P\+R\+OS tutorials are designed to show the application of C/\+C++ programming to a P\+R\+OS project. The \href{../tutorials/walkthrough/clawbot.html}{\tt Programming the Clawbot} tutorial is a great place to start, as it goes through every step of putting together a sample P\+R\+OS project. Once you are ready to branch out and create your own custom project, looking through the following tutorials is recommended\+:


\begin{DoxyItemize}
\item \href{../tutorials/general/project-structure.html}{\tt P\+R\+OS Project Structure}
\item \href{../tutorials/walkthrough/uploading.html}{\tt Uploading Code}
\item \href{../tutorials/general/debugging.html}{\tt Debugging}
\item \href{./faq.html}{\tt Coding F\+A\+Qs}
\end{DoxyItemize}

And then you can find tutorials for specific subjects from \href{../tutorials/topical/adi.html}{\tt the A\+DI} to \href{../tutorials/topical/multitasking.html}{\tt tasks and multithreading} as well.

V\+EX U team Q\+U\+E\+EN has also created a series of You\+Tube videos about C++, object-\/oriented programming, and other relevant programming topics in the context of V\+EX robots; you can find it \href{https://www.youtube.com/playlist?list=PLxt0dHFRDpQhy24IL1wAniVq3xf8N7QAV}{\tt here}.

\subsection*{How do I share my code with other people?}

In order to track changes to your code, as well as to help share your code with other people, you can use a version control system like Git in conjunction with a service like \href{https://github.com}{\tt Git\+Hub}. If you want to share smaller parts of your code with other people in order to get help (e.\+g. through the software channel on Discord), you can use something like \href{https://gist.github.com}{\tt Git\+Hub Gists} or \href{https://pastebin.com}{\tt Pastebin}.

For more information on how to use Git, including the Git C\+LI tools and setting up a respository on Git\+Hub, check out the \href{https://help.github.com}{\tt Git\+Hub help pages}. 