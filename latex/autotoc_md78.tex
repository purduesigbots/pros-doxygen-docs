\begin{quote}
{\bfseries note}

For a full list of functions for interacting with the V5 Vision Sensor, see its \+: \href{../../api/c/vision.html}{\tt C A\+PI} and \href{../../api/cpp/vision.html}{\tt C++ A\+PI}. \end{quote}


The first step to using the vision sensor is setting color signatures for the sensor to recognize as objects. This is done through the V5 Vision Utility program.

The primary function of the vision sensor is returning objects, or blobs of color detected by the sensor. The characteristics of an object are defined in \href{../../api/c/vision.html#vision_object_s_t}{\tt vision\+\_\+object\+\_\+s\+\_\+t}.

The simplest way to interact with the vision sensor is to get an object by its size. 0 is the largest object detected by the sensor.

If you have multiple signatures saved to the vision signature, you will most likely want to only look for objects of a particular signature. The {\ttfamily get\+\_\+by\+\_\+sig()} function implements this functionality.

Each returned object from the vision sensor comes with a set of coordinates telling where the object was found in the vision sensor\textquotesingle{}s field of view. The default behavior is to return the coordinates as a function of distance from the top left corner of the field of view -\/ so positive y is downward and positive x is right. With the P\+R\+OS A\+PI, you can change this behavior so that the center of the Field Of View is the (0,0) point for object coordinates. Positive y is still downward and positive x is still right, but negative y is upward of center and negative x is left of center in this configuration.

In P\+R\+OS Kernel 3.\+1.\+4 and earlier, the vision sensor exposure parameter was in the range \mbox{[}0,58\mbox{]}. In P\+R\+OS Kernel 3.\+1.\+5 and newer, the parameter is scaled to be in the range \mbox{[}0,150\mbox{]} to match the Vision Sensor utility. As a result, there is a loss of information in this translation since multiple integers on the scale \mbox{[}0,150\mbox{]} map to the scale \mbox{[}0,58\mbox{]}. 