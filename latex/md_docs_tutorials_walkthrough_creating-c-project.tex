By default, a new P\+R\+OS project contains C++ source files and the C++ A\+PI. If you would prefer to program in C instead, change the extension of the source files (prior to P\+R\+OS kernel 3.\+2.\+0\+: {\ttfamily initialize.\+cpp}, {\ttfamily autonomous.\+cpp}, and {\ttfamily opcontrol.\+cpp}; after P\+R\+OS kernel 3.\+2.\+0\+: {\ttfamily main.\+cpp}) to {\ttfamily .c}.

\begin{DoxyWarning}{Warning}
Do not change any of the P\+R\+OS header files in this process. That will cause the wrong files to be included in your project, and will likely prevent compilation. Only modify the extensions of the {\ttfamily .cpp} files.
\end{DoxyWarning}
This will compile your P\+R\+OS project as C code, and will give you access to the \href{../../api/c/index.html}{\tt C A\+PI}.

If you\textquotesingle{}re interested in combining C and C++, you should read through the \href{../general/combining-c-cpp.html}{\tt Combining C and C++ tutorial}. Please note that it is typically a complicated matter to combine the two, and rarely a good idea. 