\begin{quote}
{\bfseries note}

For a full list of functions for interacting with the A\+DI, see its \+: \href{../../api/c/adi.html}{\tt C A\+PI} and \href{../../api/cpp/adi.html}{\tt C++ A\+PI}.

\end{quote}


While computers, microcontrollers, and other devices that interface with V\+EX robots are digital systems, most of the real world operates as analog components, where a range of possible values exist instead of simply an arrangement of 1s and 0s. Analog sensors like potentiometers and line trackers are used to communicate with these analog real-\/world systems. These sensors return a number within a preset range of values in accordance with their input, as opposed to a digit sensor which simply returns an on or off state.

To take these analog inputs and convert them to information that the Cortex can actually use, A\+D\+Cs (Analog to Digital Converters) are used on each of the Analog In ports to convert the analog input signals (varying voltage signals) to 12 bit integers. As a result, the range of all analog sensors when used with the Cortex is 0 to 4095 (the range of a 12 bit unsigned integer).

As with all A\+DI sensors, the first step to using the sensor is to set the configuration for its A\+DI port.

Additionally, it is often worthwhile to calibrate analog sensors before using them in the {\ttfamily \hyperlink{main_8h_a9efe22aaead3a5e936b5df459de02eba}{initialize()}} function. The \href{../../api/c/adi.html#adi-analog-calibrate}{\tt analog\+\_\+calibrate} function collects approximately 500 data samples over a period of half a second and returns the average value received over the sampling period. This average value can be used to account for variations like ambient light for line trackers.

Potentiometers measure angular position and can be used to determine the direction of rotation of its input. Potentiometers are best used in applications such as lifts where the sensor is not at risk of being rotated beyond its 250-\/degree physical constraint. Potentiometers typically do not need to be calibrated, although it may be desired as it helps account for possible shifting in the potentiometer mounting and to find the actual range of the potentiometer due to its mechanical stops as that range may be closer to 5-\/4090 instead of 0-\/4095. If the potentiometer is not calibrated, the \href{../../api/c/adi.html#adi-analog-read}{\tt analog\+\_\+read} function may be used to obtain the raw input value of the potentiometer. If the sensor was calibrated, the \href{../../api/c/adi.html#adi-analog-read-calibrated}{\tt analog\+\_\+read\+\_\+calibrated} function should be used, as it will account for the sensor\textquotesingle{}s calibration and return more accurate results. The input to both of these functions is the channel number of the sensor, and an integer is returned.

Thus an example of use on a lift would look like\+:

V\+EX Line Trackers operate by measuring the amount of light reflected to the sensor and determining the existence of lines from the difference in light reflected by the white tape and the dark tiles. The Line Trackers return a value between 0 and 4095, with 0 being the lightest reading and 4095 the darkest. It is recommended that Line Trackers be calibrated to account for changes in ambient light.

An example of Line Tracker use\+:

The V\+EX Accelerometer measures acceleration on the x, y, and z axes simultaneously. Accelerometers can be used to infer velocity and displacement, but due to the error induced by such integration it is recommended that simply the acceleration data be used. By design of the V\+EX Accelerometer each axis is treated as its own analog sensors. Due to this the V\+EX Accelerometer requires three analog input ports on the Cortex.

Example accelerometer use\+:

As with all A\+DI sensors, the first step to using the sensor is to set the configuration for its A\+DI port.

From there, using a digital sensor is fairly straightforward. Digital Sensors always return a true or false (boolean) value.

Quadrature encoders can measure the rotation of the attached axle on your robot. Most common uses of this sensor type are to track distance traveled by attaching them to your robots drivetrain and monitoring how much the axle spins.

With these sensors 1 measured tick is 1 degree of revolution.

\begin{quote}
{\bfseries note}

Encoders must be plugged into the A\+DI such that the top wire \+: is in an odd numbered port (1, 3, 5, 7 or \textquotesingle{}A\textquotesingle{}, \textquotesingle{}C\textquotesingle{}, \textquotesingle{}E\textquotesingle{}, or \textquotesingle{}G\textquotesingle{}), and then the bottom wire must be in the next highest port number.

\end{quote}
Encoders are initialized as such\+:

\begin{quote}

\begin{DoxyCode}
void initialize() \{
  encoder = adi\_encoder\_init(QUAD\_TOP\_PORT, QUAD\_BOTTOM\_PORT, false);
\}
\end{DoxyCode}
 \end{quote}


And then used in the following manner\+:

Ultrasonic sensors are used in a manner that is very similar to encoders, given that they are both two-\/wire sensors.

\begin{quote}
{\bfseries note}

Ultrasonic sensors must be plugged into the A\+DI such that the P\+I\+NG wire \+: (the orange O\+U\+T\+P\+UT cable) is in an odd numbered port (1, 3, 5, 7 or \textquotesingle{}A\textquotesingle{}, \textquotesingle{}C\textquotesingle{}, \textquotesingle{}E\textquotesingle{}, or \textquotesingle{}G\textquotesingle{}), and then the E\+C\+HO wire (the yellow I\+N\+P\+UT cable) must be in the next highest port number.

\end{quote}
Ultrasonic sensors are initialized as such\+:

\begin{quote}

\begin{DoxyCode}
void initialize() \{
  ultrasonic = adi\_ultrasonic\_init(ULTRA\_PING\_PORT, ULTRA\_ECHO\_PORT);
\}
\end{DoxyCode}
 \end{quote}


And then used in the following manner\+:

Pneumatics in V\+EX provide two-\/state linear actuation. They differ from other digital sensors in that they are output signals. Therefore, the default digital sensor configuration is insufficient.

And then the pneumatics are used as such\+: 