There are currently two ways to install P\+R\+OS 3 on mac\+OS. The recommended method uses \href{https://brew.sh/}{\tt Homebrew}, and the other method involves installing components manually.

The recommended method of installing P\+R\+OS 3 for mac\+OS involves using \href{https://brew.sh/}{\tt Homebrew}.


\begin{DoxyEnumerate}
\item If you do not already have Homebrew installed, install it by following the instructions on \href{https://brew.sh}{\tt their site}. This will take a while, and may prompt you to follow some additional instructions.
\item Once you have Homebrew installed, run brew tap osx-\/cross/arm \&\& brew install arm-\/gcc-\/bin to register a repository with Homebrew that contains the toolchain used to build P\+R\+OS projects, and then install the toolchain.
\item Run brew tap purduesigbots/pros to register the P\+R\+OS Homebrew repository with Homebrew.
\item Run brew cask install pros-\/editor to install the P\+R\+OS Editor (the C\+LI will also be installed). This may also take a while.
\item That\textquotesingle{}s it! You can now start using P\+R\+OS 3.
\end{DoxyEnumerate}

\begin{DoxyNote}{Note}
If you do not want to use the P\+R\+OS Editor, and instead intend to use only the P\+R\+OS C\+LI, substitute the command in step 3 with the following\+: brew install pros-\/cli.
\end{DoxyNote}


If you don\textquotesingle{}t want to use Homebrew to install P\+R\+OS 3, you can install all the components manually.


\begin{DoxyEnumerate}
\item Download the latest version of the G\+NU Arm Embedded Toolchain for mac\+OS from \href{https://developer.arm.com/open-source/gnu-toolchain/gnu-rm/downloads}{\tt their site}.
\item Once you have downloaded the toolchain, double click the file to extract its contents.
\item Copy the contents of the gcc-\/arm-\/none-\/eabi-\/\+X-\/20\+X\+X-\/q\+X-\/update folder (where the Xs are numbers specific to the version you downloaded) to another folder, for example /usr/local/lib/pros-\/toolchain.
\item Now you will need to link the toolchain binaries to somewhere that the system will be able to find them. There are two ways to do this\+: i) (recommended, easy to update) run {\ttfamily mkdir -\/p /usr/local/bin/pros-\/toolchain \&\& ln -\/s /usr/local/lib/pros-\/toolchain/bin/$\ast$ /usr/local/bin/pros-\/toolchain} (replacing {\ttfamily /usr/local/lib/pros-\/toolchain} with the path to the folder you made in step 4 above). Finally, add {\ttfamily /usr/local/bin/pros-\/toolchain} to the end of your {\ttfamily /etc/paths} file. ii) (easier, less easy to update) simply run {\ttfamily ln -\/s /usr/local/lib/pros-\/toolchain/bin/$\ast$ /usr/local/bin}.
\end{DoxyEnumerate}


\begin{DoxyEnumerate}
\item Install Python 3.\+6 or higher from \href{http://python.org}{\tt the Python website}.
\item Install the C\+LI by downloading the latest version of the Python Wheel file (.whl) from \href{https://github.com/purduesigbots/pros-cli3/releases/latest}{\tt here}. Once downloaded, run python3 -\/m pip install/\+Downloads/pros-\/cli-\/v5\+\_\+3.\+X.\+X-\/py3-\/none-\/any.\+whl (replacing that path with the path to which you downloaded the file).
\end{DoxyEnumerate}

\begin{DoxyNote}{Note}
this section is optional if you intend to use an editor other than the P\+R\+OS Editor
\end{DoxyNote}

\begin{DoxyEnumerate}
\item Build and install cquery by following the instructions on \mbox{[}their wiki page\mbox{]}(\href{https://github.com/cquery-project/cquery/wiki/Building-cquery}{\tt https\+://github.\+com/cquery-\/project/cquery/wiki/\+Building-\/cquery}).
\item Download the pros-\/editor-\/mac.\+zip file from \href{https://github.com/purduesigbots/atom/releases/latest}{\tt our releases page}. Once downloaded, double click to extract the application, then drag the P\+R\+OS Editor.\+app file to your /\+Applications folder.
\end{DoxyEnumerate}

Minimum mac\+OS version\+: 10.\+8 Minimum Python version\+: 3.\+6

Runtime\+Error\+: Click will abort further execution because Python 3 was configured to use A\+S\+C\+II as encoding for the environment.

If you are using the P\+R\+OS Editor, open up your init script (File $>$ Init Script) and add the following two lines\+:


\begin{DoxyCode}
process.env.LANG = 'en\_US.utf-8'
process.env.LC\_ALL = 'en\_US.utf-8'
\end{DoxyCode}


If you are just using the C\+LI at the Terminal\+:


\begin{DoxyEnumerate}
\item Open up your Terminal.
\item Run cd to make sure you\textquotesingle{}re in your home directory.
\item Run touch .bash\+\_\+profile to make sure you have a shell login configuration file.
\item Edit the/.bash\+\_\+profile file in your preferred editor (you can also run open -\/e .bash\+\_\+profile to edit it in Text\+Edit), adding the following two lines at the end\+:
\end{DoxyEnumerate}


\begin{DoxyCode}
export LANG="en\_US.utf-8"
export LC\_ALL="en\_US.utf-8"
\end{DoxyCode}



\begin{DoxyEnumerate}
\item Run . .bash\+\_\+profile to reload the file for the current session.
\end{DoxyEnumerate}

/bin/sh\+: intercept-\/c++\+: command not found

\begin{DoxyNote}{Note}
This issue should be fixed for P\+R\+OS C\+LI versions $>$ 3.\+1.\+2
\end{DoxyNote}

\begin{DoxyEnumerate}
\item Check your P\+R\+OS C\+LI version by running prosv5 --version. If your version is $<$= 3.\+1.\+2, try updating first to check if that solves your problem. If not, continue with step 2.
\item Follow steps 1-\/4 listed above for those using the C\+LI only. In step 4, however, add the following line instead (replacing the Xs with the numbers found in step 1)\+:
\end{DoxyEnumerate}


\begin{DoxyCode}
export PATH="/usr/local/Cellar/pros-cli/3.X.X/libexec/bin:$PATH"
\end{DoxyCode}
 