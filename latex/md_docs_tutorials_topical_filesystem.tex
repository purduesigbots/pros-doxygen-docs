You can interact with files on the micro\+SD card (you can {\bfseries not} interact with files on the V5 brain\textquotesingle{}s flash) through standard C/\+C++ file I/O methods. For the most part, you can follow along with any standard C tutorial for file I/O and it will work with P\+R\+OS. Here are a couple of recommended tutorials\+:


\begin{DoxyItemize}
\item \href{https://www.cprogramming.com/tutorial/cfileio.html}{\tt https\+://www.\+cprogramming.\+com/tutorial/cfileio.\+html}
\item \href{https://www.tutorialspoint.com/cprogramming/c_file_io.htm}{\tt https\+://www.\+tutorialspoint.\+com/cprogramming/c\+\_\+file\+\_\+io.\+htm}
\end{DoxyItemize}

The only additional detail needed for interacting with the filesystem in P\+R\+OS is that any files on the micro\+SD card {\bfseries must} be prefaced with {\ttfamily /usd/}. A file on the micro\+SD card can be written to in the following manner\+:


\begin{DoxyCode}
FILE* usd\_file\_write = fopen("/usd/example.txt", "w");
fputs("Example text", usd\_file\_write);
fclose(usd\_file\_write);

FILE* usd\_file\_read = fopen("/usd/example.txt", "r");
char buf[50]; // This just needs to be larger than the contents of the file
fread(buf, 1, 50, usd\_file\_read); // passing 1 because a `char` is 1 byte, and 50 b/c it's the length of
       buf
printf("%s\(\backslash\)n", buf); // print the string read from the file
// Should print "Example text" to the terminal
fclose(usd\_file\_read); // always close files when you're done with them
\end{DoxyCode}


The micro\+SD card must be fat32 in order to work.

It\textquotesingle{}s also possible to interact with the serial communications ({\ttfamily stdin}, {\ttfamily stdout}, etc.) through the filesystem drivers. You can write and read from these streams in the same manner as a file, but using the four character stream identifiers.

For instance, you can write to {\ttfamily stderr} in the following manner\+:


\begin{DoxyCode}
FILE* stderr = fopen("serr", "w");
fputs("Example text", stderr);
fclose(usd\_file\_write);
\end{DoxyCode}


There are also a number of methods for controlling serial communication behavior exposed in \href{../../extended/apix.html}{\tt apix.\+h}. These methods can be accessed through the {\ttfamily \hyperlink{apix_8h_a962daefd6f45a8def6ff00802a23fbff}{serctl()}} function. At the moment two actions are supported -\/activating/deactivating the streams, and enabling/disabling \href{https://en.wikipedia.org/wiki/Consistent_Overhead_Byte_Stuffing}{\tt C\+O\+BS} . If you want to read the serial comms yourself (without using {\ttfamily pros terminal}), then you\textquotesingle{}ll want to disable C\+O\+BS. 