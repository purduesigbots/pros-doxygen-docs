\begin{quote}
{\bfseries warning}

This content is subject to change as the P\+R\+OS C\+LI for V5 is created. \end{quote}


The \href{../../api/index.html}{\tt P\+R\+OS A\+PI} provides functions like \href{http://www.cplusplus.com/reference/cstdio/printf/}{\tt printf} that allow your robot to output information to a connected serial console during operation.

\subsection*{Viewing printf output}

To view a robot\textquotesingle{}s output, there are two officially supported methods\+:


\begin{DoxyEnumerate}
\item Through the P\+R\+OS C\+LI\+:

Running {\ttfamily prosv5 terminal} on the command line will open an output stream from a robot connected over direct U\+SB connection, V\+E\+Xnet, or \href{./tutorials/topical/jinx.html}{\tt J\+I\+NX}.
\item From within Atom\+:

Click the button labeled \char`\"{}\+Open cortex serial output\char`\"{}
\end{DoxyEnumerate}



A terminal panel will open at the bottom of the screen containing the output of a connected robot.



\subsection*{Further debug info with {\ttfamily errno}}

{\ttfamily errno} is a global value that is set when any part of the P\+R\+OS kernel encounters an error. The value of {\ttfamily errno} is specific to each function, so check the function headers or the \href{../../api/index.html}{\tt A\+PI docs} for possible values and their meaning. If you think you\textquotesingle{}re encountering an error in the kernel code, check the value of {\ttfamily errno} to see what\textquotesingle{}s causing the issue.

Debugging in this manner is standard to other environments besides P\+R\+OS. For further information on using {\ttfamily errno}, see the following tutorial\+: \href{https://www.tutorialspoint.com/cprogramming/c_error_handling.htm}{\tt https\+://www.\+tutorialspoint.\+com/cprogramming/c\+\_\+error\+\_\+handling.\+htm}

\subsection*{J\+I\+NX Graphical Debugger}

J\+I\+NX offers further debugging functionality over traditional debugging through print statements. For a full explanation of J\+I\+NX\textquotesingle{}s abilities and its use, see ../topical/jinx. 