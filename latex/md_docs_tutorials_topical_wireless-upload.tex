\begin{quote}
{\bfseries note}

Wireless Upload requires at least C\+LI 3.\+1.\+4, Kernel 3.\+1.\+6, V\+E\+Xos 1.\+0.\+5, and Okapi\+Lib 3.\+3.\+12 (if used). \end{quote}


P\+R\+OS supports wireless upload to the V5 Brain via a V5 Controller. Although there are no special requirements to enable P\+R\+OS to do this, file transfer speeds are typically unacceptably slow. To make them more reasonable, P\+R\+OS has a different method of compiling your project so that the P\+R\+OS kernel and other infrequently modified code are uploaded once and only the code you regularly modify (e.\+g. opcontrol.\+cpp, autonomous.\+cpp, and initialize.\+cpp) gets transferred to the V5 wirelessly. We call the code you upload once and never change the \char`\"{}cold
image\char`\"{} and the code that changes frequently and gets uploaded every time the \char`\"{}hot image.\char`\"{}

To enable this compilation mode, you should open your project\textquotesingle{}s Makefile and edit the line\+:


\begin{DoxyCode}
# Set to 1 to enable hot/cold linking
USE\_PACKAGE:=0
\end{DoxyCode}


so that {\ttfamily U\+S\+E\+\_\+\+P\+A\+C\+K\+A\+GE\+:=1}. Re-\/build and upload your project and P\+R\+OS will automatically use the new files. P\+R\+OS will automatically detect when you\textquotesingle{}re using the same combination of libraries and decide not to re-\/upload the \char`\"{}cold\char`\"{} code. Verification that the library is present is done with filename and checksum of the file. As a result, if you use the same combination of libraries (cold image), only one copy of the cold image is uploaded to the V5.

\subsection*{Large Projects}

Projects with large codebases may still take some time to upload even with hot/cold linking. You may be able to reduce the size of your hot image by making part of your project a library so that some code is included in the cold image. To do so, edit the following lines of your project\textquotesingle{}s Makefile\+:


\begin{DoxyCode}
# Set this to 1 to add additional rules to compile your project as a PROS library template
IS\_LIBRARY:=0
# TODO: CHANGE THIS!
LIBNAME:=libbest
VERSION:=1.0.0
# EXCLUDE\_SRC\_FROM\_LIB= $(SRCDIR)/unpublishedfile.c
\end{DoxyCode}


Change {\ttfamily I\+S\+\_\+\+L\+I\+B\+R\+A\+RY} to {\ttfamily 1} and rename the library to something of your choosing. We recommend {\ttfamily lib$<$robot name$>$} or {\ttfamily lib$<$team number$>$} (e.\+g. {\ttfamily lib7701}) to be unique enough to avoid cause naming conflicts with other cold images. When compiling, P\+R\+OS will include this library as part of the cold image. Your library should only contain files you infrequently change so that you do not have to frequently upload the cold image. By default, the Makefile is set up to exclude your project\textquotesingle{}s opcontrol.\+cpp, autonomous.\+cpp, and initialize.\+cpp files. If you change other files frequently, you can add lines like {\ttfamily E\+X\+C\+L\+U\+D\+E\+\_\+\+S\+R\+C\+\_\+\+F\+R\+O\+M\+\_\+\+L\+I\+B+=/myfile.c} as needed.

{\bfseries An important caveat} is that code that goes into the cold image is {\bfseries not} able to link against anything (call functions or use variables) in the hot image. Doing so will result in a linker error -\/ you will be told what\textquotesingle{}s trying to refer to what in the hot image\+: \begin{DoxyVerb}bin/libtheseus.a(lcdselector.cpp.o):(.data.inits+0x0): undefined reference to `auton::allianceInit(auton::color)'
\end{DoxyVerb}


You should refactor your code so that only \char`\"{}hot\char`\"{} code calls \char`\"{}cold\char`\"{} code or include the culprit file in the hot image.

An example of a modified Makefile\textquotesingle{}s relevant lines is shown below\+:


\begin{DoxyCode}
# Set to 1 to enable hot/cold linking
USE\_PACKAGE=1

# Set this to 1 to add additional rules to compile your project as a PROS library template
IS\_LIBRARY:=1
LIBNAME:=libtheseus
VERSION:=1.0.0
# this line excludes opcontrol.c and similar files
EXCLUDE\_SRC\_FROM\_LIB+=$(foreach file, $(SRCDIR)/opcontrol $(SRCDIR)/initialize
       $(SRCDIR)/autonomous,$(foreach cext,$(CEXTS),$(file).$(cext)) $(foreach cxxext,$(CXXEXTS),$(file).$(cxxext)))
EXCLUDE\_SRC\_FROM\_LIB+=$(SRCDIR)/scripts             # exclude any files in the src/scripts directory
EXCLUDE\_SRC\_FROM\_LIB+=$(SRCDIR)/lcdselector.cpp     # exclude src/lcdselector.cpp
\end{DoxyCode}


\subsection*{Hot/\+Cold Linking Troubleshooting}

Since hot/cold linking involves ensuring two compiled programs interact consistently, there may be additional runtime issues running in this mode. This section serves as a guide for debugging these sorts of issues.

Generally, if the program appears to be running correctly (i.\+e. the screen shows), then the compilation mode worked correctly and the error you\textquotesingle{}re experiencing is likely related to your code\textquotesingle{}s logic. If your code\textquotesingle{}s logic runs correctly when not using hot/cold linking, contact us so we may assist in troubleshooting.

If you see a black scren, then P\+R\+OS did not boot correctly.


\begin{DoxyItemize}
\item A global constructor is in an infinite loop or raised an exception.
\item See also troubleshooting steps below
\end{DoxyItemize}

If you see a grey screen, then P\+R\+OS booted correctly, but is not running your hot image.


\begin{DoxyItemize}
\item Delete all user programs, perform a clean build, and upload
\end{DoxyItemize}

If you\textquotesingle{}re having issues, contact us so we may assist in troubleshooting. 